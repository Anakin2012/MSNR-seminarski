% !TEX encoding = UTF-8 Unicode
\documentclass[a4paper]{article}

\usepackage{color}
\usepackage{url}
\usepackage[T2A]{fontenc} % enable Cyrillic fonts
\usepackage[utf8]{inputenc} % make weird characters work
\usepackage{graphicx}

\usepackage[english,serbian]{babel}
%\usepackage[english,serbianc]{babel} %ukljuciti babel sa ovim opcijama, umesto gornjim, ukoliko se koristi cirilica

\usepackage[unicode]{hyperref}
\hypersetup{colorlinks,citecolor=green,filecolor=green,linkcolor=blue,urlcolor=blue}

\usepackage{listings}

\newcommand\todos[1]{\textcolor{red}{#1}}

%\newtheorem{primer}{Пример}[section] %ćirilični primer
\newtheorem{primer}{Primer}[section]

\definecolor{mygreen}{rgb}{0,0.6,0}
\definecolor{mygray}{rgb}{0.5,0.5,0.5}
\definecolor{mymauve}{rgb}{0.58,0,0.82}

\lstset{ 
  backgroundcolor=\color{white},   % choose the background color; you must add \usepackage{color} or \usepackage{xcolor}; should come as last argument
  basicstyle=\footnotesize,        % the size of the fonts that are used for the code
  breakatwhitespace=false,         % sets if automatic breaks should only happen at whitespace
  breaklines=true,                 % sets automatic line breaking
  captionpos=b,                    % sets the caption-position to bottom
  commentstyle=\color{mygreen},    % comment style
  deletekeywords={...},            % if you want to delete keywords from the given language
  escapeinside={\%*}{*)},          % if you want to add LaTeX within your code
  extendedchars=true,              % lets you use non-ASCII characters; for 8-bits encodings only, does not work with UTF-8
  firstnumber=1000,                % start line enumeration with line 1000
  frame=single,	                   % adds a frame around the code
  keepspaces=true,                 % keeps spaces in text, useful for keeping indentation of code (possibly needs columns=flexible)
  keywordstyle=\color{blue},       % keyword style
  language=Python,                 % the language of the code
  morekeywords={*,...},            % if you want to add more keywords to the set
  numbers=left,                    % where to put the line-numbers; possible values are (none, left, right)
  numbersep=5pt,                   % how far the line-numbers are from the code
  numberstyle=\tiny\color{mygray}, % the style that is used for the line-numbers
  rulecolor=\color{black},         % if not set, the frame-color may be changed on line-breaks within not-black text (e.g. comments (green here))
  showspaces=false,                % show spaces everywhere adding particular underscores; it overrides 'showstringspaces'
  showstringspaces=false,          % underline spaces within strings only
  showtabs=false,                  % show tabs within strings adding particular underscores
  stepnumber=2,                    % the step between two line-numbers. If it's 1, each line will be numbered
  stringstyle=\color{mymauve},     % string literal style
  tabsize=2,	                   % sets default tabsize to 2 spaces
  title=\lstname                   % show the filename of files included with \lstinputlisting; also try caption instead of title
}

\begin{document}

\title{Privatnost na Internetu\\ \small{Seminarski rad u okviru kursa\\Metodologija stručnog i naučnog rada\\ Matematički fakultet}}

\author{Đaković Branko, Krčmarević Mladen,\\Petrović Ana, Spasojević Đorđe\\ brankodjakovic08@gmail.com, mladenk@twodesperados.com,\\ pana.petrovic@gmail.com, djordje.spasojevic1996@gmail.com}

%\date{9.~april 2015.}

\maketitle

\abstract{ 
\todos{TODO napisati apstrakt (na kraju)}\\
Ovde pišemo opis ukratko svega što smo radili. Na ovoj strani mora da bude naslov, ovaj apstrakt i sadržaj - SVE MORA DA STANE}

\tableofcontents

\newpage

\section{Uvod}
\label{sec:uvod}
U martu 2018. godine bivši zaposleni firme Kembridž Analitika (eng.~{\em Cambridge Analytica}) je otkrio javnosti da je ova firma kupila od društvene mreže Fejsbuk privatne podatke barem 50 miliona korisnika bez njihove saglasnosti \cite{guardian}. Podaci su prikupljeni uz pomoć aplikacije na Fejsbuku koja je bila namenjena prikupljanju podataka u akadamske svrhe. Međutim, ova aplikacija je, osim za one korisnike koji su na to pristali, uzimala podatke i svih njihovih prijatelja. Na osnovu ove velike količine podataka pravljeni su profili koji su služili za ciljano oglašavanje tokom predsedničke kampanje Donalda Trampa, kao i tokom perioda referenduma o članstvu Ujedinjenog Kraljevstva u Evropskoj uniji. Politika platforme Fejsbuk izričito zabranjuje prodaju ili korišćenje podataka korisnika u svrhe oglašavanja, što je učinilo ovaj slučaj jednim od najvećih slučaja kršenja privatnosti podataka na Internetu. Iako je Trampov kabinet porekao korišćenje ovih podataka u cilju pribavljanja glasača, ova vest je otvorila diskusije o tome ko i u kolikoj meri ima pristup našim podacima na Internetu.
\par Uzimajući u obzir da društvene mreže kao što je Instagram imaju milijardu korisnika na mesečnom nivou \cite{instagram} , količina privatnih podataka koje oni ostavljaju na ovoj platformi, a koji se kreću od uzrasta i lokacije, do interesovanja i hobija, je ogromna. Uz to, ove kompanije najčešće ne traže dozvolu da čuvaju podatke korisnika, ili kada je traže, to rade na netransparentan način. Povrh svega toga, pojedinci najčešće nisu ni svesni da sve što ostave na Internetu, kasnije može biti iskorišćeno na načine koji bi u nekim slučajevima mogli da idu i na njihovu štetu.
\todos{TODO: U prvoj sekciji bla bla, drugoj, trecoj .... bla bla bla (kad oderedimo struktru)}


\section{Šta je privatnost?}
\label{sec:prvoPoglavlje}
Pravo na privatnost se često smatra za jedno od najjasnijih i najvrednovanijih ljudskih prava, ali je privatnost koncept koji je teško definisati \cite{solove}. Jedan od problema u definisanju privatnosti leži u širini i neodređenosti ovog pojma, koji obuhvata zaštitu od ispitivanja i nedozvoljenih pretraga, kontrolu nad sopstvenim telom i privatnim informacijama, pa čak i pravo na slobodu mišljenja i govora. Ključni pojam u raspravama o tome šta je privatnost je koncept pristupa, bilo u kontekstu fizičke bliskosti nekoj osobi, bilo u kontekstu posedovanja znanja o toj osobi \cite{ethics}. Pristup podrazumeva da jedna osoba ima pravo da ograniči ili zabrani pristup sebi u najširem smislu te reči, dok sa druge strane, on takođe podrazumeva i pravo drugih da ostvare pristup određenoj osobi. Upravo je sukob između ova dva ono u čemu leži suština privatnosti. Taj sukob se najviše reflektuje u suprotstavljanju  privatnog i javnog, i onoga što bi trebalo da bude privatno i javno. Sa jedne strane, svaki pojedinac ima pravo da ograniči informacije o sebi koje su deo javnog znanja, međutim, preveliko ograničenje pristupa može dovesti do zloupotrebe i do loših posledica po društvo. Upravo zato privatnost predstavlja ključan problem slobode i demokratije \cite{solove}. Zbog toga ni ne čudi što se ogroman broj zakona i ustavnih prava odnosi upravo na privatnost. Pri tome, često se sprovođenje ovih zakona i odredbi dovodi u pitanje usled nedefinisanosti pojma. jedan od mogućih načina da definišemo privatnost je da je konceptualizujemo kao društveni ugovor koji dozvoljava pojedincu da ima određen nivo kontrole nad time ko i u kojoj meri ima pristup ne samo njegovim podacima, već i njegovom telu \cite{ethics}.

\subsection{Privatnost na Internetu}
\label{subsec:privatnostNaInternetu}

Sa razvojem Interneta, problem shvatanja privatnosti se dodatno komplikuje i činjenica da veliku većinu dana provodimo na Internetu donosi nove aspekte problemu privatnosti. Svakog dana, korišćenjem interneta za različite potrebe, svaka osoba ostavlja za sobom elektronski trag svojih aktivnosti, koje mogu biti takve da otkrivaju i identitet pojedinca. Uz to, ostavljanje privatnih podataka može imati i određene prednosti, kao što su personalizovane poruke ili na primer, popusti, te je pojedinac suočen sa odlukom da li da propusti ove prednosti, ali očuva privatnost ili da ostavi svoje podatke na internetu, koji kasnije mogu biti zloupotrebljeni \cite{renata}.
\par Povrh ovih informacija koje sami voljno ili nevoljno ostavljamo, danas su i informacije koje su deo javnih informacija (kao što su podaci o rođenju ili bračnom statusu pojedinca) kompjuterizovane, što ih čini daleko dostupnijim. Sa jedne strane, ovo ima svoju jasnu prednost, ali ovo podrazumeva da su i ove informacije daleko pristupačnije i onima kojima možda ne bismo želeli da budu \cite{ethics}. Kapacitet prikupljanja podataka svakim danom sve više raste, što dovodi do mnogo lakšeg prepoznavanja pojedinca i korišćenja podataka o njemu u različite svrhe. Iako često mi sami odlučujemo da podelimo neke privatne informacije na internetu, često pojedinac nije ni svestan kada daje dozvolu nekoj privatnoj kompaniji da koristi njegove informacije i najčešće su pojedinci zbunjeni oko toga šta su tačno njihova prava na privatnost na internetu \cite{renata}.

\subsection{Percepcija privatnosti}
\label{subsec:percepcija}

Bitan aspekt razmatranja privatnosti na internetu je koncept percipirane privatnosti. Odnosno, problem privatnosti ne ostaje samo na tome da li ona postoji ili ne, već se širi i na to da li pojedinac vidi svoju privatnost kao problem i ako da, u kojoj meri. Percepcija privatnosti dalje određuje ponašanje pojedinca na internetu, utoliko što će pojedinci biti spremniji da koriste određene veb stranice ukoliko smatraju da je njihova privatnost zaštićena i obrnuto. Povrh toga, pokazuje se da, kada postoji uverenost da je privatnost pojedinca zaštićena, oni su spremni da je se vrlo lako odreknu, radi ispunjenja nekih potreba (kao što je na primer kupovina).
\par Percepciju privatnosti na internetu oblikuju različiti faktori, od kojih se neki vezuju za samog pojedinca, neki za same stranice koje korisnici posećuju, a neki za datu situaciju. Tako, shvatanje privatnosti može zavisiti od pola ili uzrasta, ali i obrazovanja pojedinca, prethodnog iskustva na internetu kao i od toga da li je osoba prethodno iskusila narušavanje privatnosti na internetu. Isto tako, to da li će pojedinac biti zabrinut za svoju privatnost na određenom sajtu može zavisiti od poznatosti brenda tog sajta, opaženog integriteta ili opaženog rizika tog sajta. Na kraju, to kako percipiramo trenutnu pretnju po privatnost može da zavisi i od toga kakvo je poklapanje između traženih informacija i usluge koju dobijamo i da li nam je ona smislena, ili od trenutne percepcije toga koliko je određeni podatak koji ostavljamo na internetu osetljive prirode \cite{renata}. Dakle, percepcija privatnosti na internetu je širok problem, koji je pod uticajem mnogih faktora, a koji povratno utiče na ponašanje svakog pojedinca na internetu, koje ponovo dovodi do narušavanja ili očuvanja privatnosti na internetu, te je stoga ovaj problem veoma relevantan za razumevanje privatnosti na internetu danas.
  

\section{Država i privatnost(\todos{promeniti naslov})}	
\label{sec:drugoPoglavlje}
Kada se govori o privatnosti na internetu, treba pomenuti i odnos države i državnih institucija prema privatnosti pojedinca. Kroz istoriju je bio čest slučaj da su države narušavale privatnost pojedinaca kako bi ostvarile određene interese, rešile probleme ili u borbi protiv terorizma. Slučajeva u kojima su američka administracija i službe kršile privatnost građana ima puno, od prodaje podataka službenika tajnih službi novinarima i privatnim detektivima, do ilegalnog prisluškivanja američkih i stranih državljana od strane FBI. \par Neki od tih slučajeva su zloupotreba popisnih spiskova od strane američke vojske u Prvom i Drugom svetskom ratu, korišćenje policijskih dronova, kao i televizijske kamere zatvorenog kruga (eng.~{\em closed-circuit television - CCTV}) - stanovnik Velike Britanije bude uhvaćen na kameri u proseku 300 puta dnevno \cite{ethics, london}. Prisluškivanje razgovora i postavljanje bubica je aktom američkog kongresa iz 1934. godine zabranjeno bez sudskog naloga. Međutim, FBI je nastavio to da radi ilegalno, čak i tokom Drugog svetskog rada uz dozvolu predsednika Ruzvelta \cite{ruzvelt}.Nakon rata, FBI, NSA i druge bezbednosne službe nastavile su sa kršenjem zakona o privatnosti pojedinaca, što su kasnije i proširili na druge vidove komunikacije, pa i na internet.

\subsection{Legalno narušavanje privatnosti na internetu}
\label{subsec:zakoni}
Usled pretnje po bezbednost SAD-a, donošeni su zakoni koji su službama dali veća ovlašćenja i dozvoljen im je upad u privatnost pojedinaca. U narednim odeljcima, biće navedeni neki zakoni kojima administracija dozvoljava narušavanje privatnosti na internetu.

\subsubsection{Zakon o nadzoru stranih službi}
Američka administracija je 1978. godine donela zakon o nadzoru stranih službi (eng.~{\em Foreign Intelligence Surveillance Act - FISA}),  kojim se dozvoljava tajni nadzor inostranih vlada i njihovih službi. Ovim zakonom američki predsednik je mogao da odobri elektronski nadzor stranih državljana na jednu godinu, pod uslovom da se time ne krši privatnost državljana Amerike. U suprotnom, administracija bi morala da dobije sudski nalog za prisluškivanje. Nakon što je 2013. godine Edvard Snouden, bivši zaposleni u NSA i FBI, obelodanio na hiljade tajnih dokumenata američkih tajnih službi, među kojima se našao i projekat PRISM, koji je dozvoljavao NSA pristup svim serverima i informacijama, pa čak i nadzor video poziva bez sudskog naloga. U ovaj tajni program su bile uključene sve velike kompanije, pa je tako NSA imao pristup serverima Majkrosofta, Jahua, Gugla, Fejsbuka, Jutjuba i Epla \cite{kompanije}. U odeljku \ref{subsec:prism} detaljno će biti obrađen projekat PRISM.

\subsubsection{Zakon o skladištenoj komunikaciji}
Zakon o skladištenoj komunikaciji (eng. ~{\em Stored Communication Act}) predstavlja deo zakona o privatnosti pri elektronskoj komunikaciji (eng. ~{\em Electronic Communication Privacy Act}) iz 1986. godine i odnosi se na privatnost kolekcija elektronske pošte. Po ovom zakonu administraciji nije potreban sudski nalog kako bi od dobavljača internet usluga (eng. ~{\em Internet provider})  dobila mejlove starije od 180 dana. Problem sa ovim zakonom nastaje usled činjenice da sve više korisnika koriste klaudove internet provajdera, tako da sada nije jedina funkcija provajdera samo prenos elektronske pošte, već sve više ljudi koriste servere internet provajdera da čuvaju stvari koje bi inače čuvali na privatnim računarima. Usled proširenja skladištenog prostora i funkcija provajderskih servera, skoro pedeset kompanija i organizacija koje smatraju da administracija ne bi smela da dobavlja privatne informacije korisnika sa klauda bez sudskog naloga, udružilo se u organizaciju pod nazivom Digital Due Process, kako bi zahtevali od administracije da unapredi ovaj zakon \cite{ddp}.



\subsection{PRISM}
\label{subsec:prism}


\todos{ djoletov drugi deo}






\section{Zaključak}
\label{sec:zakljucak}

Ovde pišem zaključak. 
Ovde pišem zaključak. 
Ovde pišem zaključak. 
Ovde pišem zaključak. 
Ovde pišem zaključak. 
Ovde pišem zaključak. 
Ovde pišem zaključak. 
Ovde pišem zaključak. 
Ovde pišem zaključak. 
Ovde pišem zaključak. 
Ovde pišem zaključak. 
Ovde pišem zaključak. 


\addcontentsline{toc}{section}{Literatura}
\appendix
\bibliography{seminarski} 
\bibliographystyle{plain}

\appendix
\section{Dodatak}
Ovde pišem dodatne stvari, ukoliko za time ima potrebe.
Ovde pišem dodatne stvari, ukoliko za time ima potrebe.
Ovde pišem dodatne stvari, ukoliko za time ima potrebe.
Ovde pišem dodatne stvari, ukoliko za time ima potrebe.
Ovde pišem dodatne stvari, ukoliko za time ima potrebe.


\end{document}
