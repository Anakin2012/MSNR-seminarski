% !TEX encoding = UTF-8 Unicode
\documentclass[a4paper]{article}

\usepackage{color}
\usepackage{url}
\usepackage[T2A]{fontenc} % enable Cyrillic fonts
\usepackage[utf8]{inputenc} % make weird characters work
\usepackage{graphicx}

\usepackage[english,serbian]{babel}
%\usepackage[english,serbianc]{babel} %ukljuciti babel sa ovim opcijama, umesto gornjim, ukoliko se koristi cirilica

\usepackage[unicode]{hyperref}
\hypersetup{colorlinks,citecolor=green,filecolor=green,linkcolor=blue,urlcolor=blue}

\usepackage{listings}

\newcommand\todos[1]{\textcolor{red}{#1}}

%\newtheorem{primer}{Пример}[section] %ćirilični primer
\newtheorem{primer}{Primer}[section]

\definecolor{mygreen}{rgb}{0,0.6,0}
\definecolor{mygray}{rgb}{0.5,0.5,0.5}
\definecolor{mymauve}{rgb}{0.58,0,0.82}

\lstset{ 
  backgroundcolor=\color{white},   % choose the background color; you must add \usepackage{color} or \usepackage{xcolor}; should come as last argument
  basicstyle=\footnotesize,        % the size of the fonts that are used for the code
  breakatwhitespace=false,         % sets if automatic breaks should only happen at whitespace
  breaklines=true,                 % sets automatic line breaking
  captionpos=b,                    % sets the caption-position to bottom
  commentstyle=\color{mygreen},    % comment style
  deletekeywords={...},            % if you want to delete keywords from the given language
  escapeinside={\%*}{*)},          % if you want to add LaTeX within your code
  extendedchars=true,              % lets you use non-ASCII characters; for 8-bits encodings only, does not work with UTF-8
  firstnumber=1000,                % start line enumeration with line 1000
  frame=single,	                   % adds a frame around the code
  keepspaces=true,                 % keeps spaces in text, useful for keeping indentation of code (possibly needs columns=flexible)
  keywordstyle=\color{blue},       % keyword style
  language=Python,                 % the language of the code
  morekeywords={*,...},            % if you want to add more keywords to the set
  numbers=left,                    % where to put the line-numbers; possible values are (none, left, right)
  numbersep=5pt,                   % how far the line-numbers are from the code
  numberstyle=\tiny\color{mygray}, % the style that is used for the line-numbers
  rulecolor=\color{black},         % if not set, the frame-color may be changed on line-breaks within not-black text (e.g. comments (green here))
  showspaces=false,                % show spaces everywhere adding particular underscores; it overrides 'showstringspaces'
  showstringspaces=false,          % underline spaces within strings only
  showtabs=false,                  % show tabs within strings adding particular underscores
  stepnumber=2,                    % the step between two line-numbers. If it's 1, each line will be numbered
  stringstyle=\color{mymauve},     % string literal style
  tabsize=2,	                   % sets default tabsize to 2 spaces
  title=\lstname                   % show the filename of files included with \lstinputlisting; also try caption instead of title
}

\begin{document}

\title{Privatnost na Internetu\\ \small{Seminarski rad u okviru kursa\\Metodologija stručnog i naučnog rada\\ Matematički fakultet}}

\author{Branko Đaković, Mladen Krčmarević,\\ Ana Petrović, Đorđe Spasojević\\ brankodjakovic08@gmail.com, mladenk@twodesperados.com,\\ pana.petrovic@gmail.com, djordje.spasojevic1996@gmail.com}

%\date{9.~april 2015.}

\maketitle

\abstract{ 
\todos{TODO napisati apstrakt (na kraju)}\\
Ovde pišemo opis ukratko svega što smo radili. Na ovoj strani mora da bude naslov, ovaj apstrakt i sadržaj - SVE MORA DA STANE}

\tableofcontents

\newpage

\section{Uvod}
\label{sec:uvod}
U martu 2018. godine bivši zaposleni firme Kembridž Analitika (eng.~{\em Cambridge Analytica}) je otkrio javnosti da je ova firma kupila od društvene mreže Fejsbuk privatne podatke barem 50 miliona korisnika bez njihove saglasnosti \cite{guardian}. Podaci su prikupljeni uz pomoć aplikacije na Fejsbuku koja je bila namenjena prikupljanju podataka u akadamske svrhe. Međutim, ova aplikacija je, osim za one korisnike koji su na to pristali, uzimala podatke i svih njihovih prijatelja. Na osnovu ove velike količine podataka pravljeni su profili koji su služili za ciljano oglašavanje tokom predsedničke kampanje Donalda Trampa, kao i tokom perioda referenduma o članstvu Ujedinjenog Kraljevstva u Evropskoj uniji. Politika platforme Fejsbuk izričito zabranjuje prodaju ili korišćenje podataka korisnika u svrhe oglašavanja, što je učinilo ovaj slučaj jednim od najvećih slučaja kršenja privatnosti podataka na Internetu. Iako je Trampov kabinet porekao korišćenje ovih podataka u cilju pribavljanja glasača, ova vest je otvorila diskusije o tome ko i u kolikoj meri ima pristup našim podacima na Internetu. Uzimajući u obzir da društvene mreže kao što je Instagram imaju milijardu korisnika na mesečnom nivou \cite{instagram} , količina privatnih podataka koje oni ostavljaju na ovoj platformi, a koji se kreću od uzrasta i lokacije, do interesovanja i hobija, je ogromna. Uz to, ove kompanije najčešće ne traže dozvolu da čuvaju podatke korisnika, ili kada je traže, to rade na netransparentan način. Povrh svega toga, pojedinci najčešće nisu ni svesni da sve što ostave na Internetu, kasnije može biti iskorišćeno na načine koji bi u nekim slučajevima mogli da idu i na njihovu štetu.
\todos{TODO: U ovom radu pokrićemo teme .... bla bla bla (kad oderedimo struktru)}


\section{Prvo poglavlje}
Tekst...


\section{Drugo poglavlje}	
\label{sec:drugo}

Tekst... 
 
\begin{primer}
U odeljku \ref{sec:naslov1} precizirani su osnovni pojmovi, dok su zaključci dati u odeljku \ref{sec:zakljucak}.
\end{primer}


\section{Treće poglavlje}
\label{trece}



\section{Prvi naslov}
\label{sec:naslov1}


Ovde pišem tekst. 
Ovde pišem tekst. 
Ovde pišem tekst. 
Ovde pišem tekst. 
Ovde pišem tekst. 
Ovde pišem tekst. 
Ovde pišem tekst. 
Ovde pišem tekst. 


\subsection{Prvi podnaslov}
\label{subsec:podnaslov1}

Ovde pišem tekst. 
Ovde pišem tekst. 
Ovde pišem tekst. 
Ovde pišem tekst. 
Ovde pišem tekst. 
Ovde pišem tekst. 
Ovde pišem tekst. 

\subsection{Drugi podnaslov}
\label{subsec:podnaslov2}

Ovde pišem tekst. 
Ovde pišem tekst. 
Ovde pišem tekst. 
Ovde pišem tekst. 
Ovde pišem tekst. 
Ovde pišem tekst. 


\subsection{... podnaslov}
\label{subsec:podnaslovN}

Ovde pišem tekst. 
Ovde pišem tekst. 
Ovde pišem tekst. 
Ovde pišem tekst. 
Ovde pišem tekst. 
Ovde pišem tekst. 

\section{n-ti naslov}
\label{sec:naslovN}

Ovde pišem tekst. 
Ovde pišem tekst. 
Ovde pišem tekst. 
Ovde pišem tekst. 
Ovde pišem tekst. 

\subsection{... podnaslov}
\label{subsec:podnaslovK}

Ovde pišem tekst. 
Ovde pišem tekst. 
Ovde pišem tekst. 
Ovde pišem tekst. 
Ovde pišem tekst. 

\subsection{... podnaslov}
\label{subsec:podnaslovM}

Ovde pišem tekst. 
Ovde pišem tekst. 
Ovde pišem tekst. 
Ovde pišem tekst. 
Ovde pišem tekst. 


\section{Zaključak}
\label{sec:zakljucak}

Ovde pišem zaključak. 
Ovde pišem zaključak. 
Ovde pišem zaključak. 
Ovde pišem zaključak. 
Ovde pišem zaključak. 
Ovde pišem zaključak. 
Ovde pišem zaključak. 
Ovde pišem zaključak. 
Ovde pišem zaključak. 
Ovde pišem zaključak. 
Ovde pišem zaključak. 
Ovde pišem zaključak. 


\addcontentsline{toc}{section}{Literatura}
\appendix
\bibliography{seminarski} 
\bibliographystyle{plain}

\appendix
\section{Dodatak}
Ovde pišem dodatne stvari, ukoliko za time ima potrebe.
Ovde pišem dodatne stvari, ukoliko za time ima potrebe.
Ovde pišem dodatne stvari, ukoliko za time ima potrebe.
Ovde pišem dodatne stvari, ukoliko za time ima potrebe.
Ovde pišem dodatne stvari, ukoliko za time ima potrebe.


\end{document}
