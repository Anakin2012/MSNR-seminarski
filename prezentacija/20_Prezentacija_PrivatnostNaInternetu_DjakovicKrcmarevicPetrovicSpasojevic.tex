\documentclass{beamer}

\mode<presentation>{
\usetheme{Darmstadt}
\setbeamercovered{transparent}
\usecolortheme{lsc}
}
%Darmstadt najbolji
%Hannover ok sa strane
\mode<handout>{
  % tema simples para ser impresso
  \usepackage[bar]{beamerthemetree}
  % Colocando um fundo cinza quando for gerar transparências para serem impressas
  % mais de uma transparência por página
  \beamertemplatesolidbackgroundcolor{black!5}
}
\usepackage[dvipsnames]{}
\usepackage{amsmath,amssymb}
\usepackage[brazil]{varioref}
\usepackage[english,brazil]{babel}
\usepackage[utf8]{inputenc}
%\usepackage[latin1]{inputenc}
\usepackage{graphicx}
\usepackage{listings}
\usepackage{url}
\usepackage{colortbl}

\definecolor{aqua}{RGB}{50, 135, 140}

\beamertemplatetransparentcovereddynamic

\title[Privatnost i njena zaštita u digitalnom dobu]{Privatnost i njena zaštita u digitalnom dobu}
\author[]{%
  Đaković Branko, Krčmarević Mladen, Petrović Ana\inst{} \\
   \inst{}}
  \institute[]{
  \inst{}%
    Matematički fakultet\\~\\
    brankodjakovic08@gmail.com, mladenk@twodesperados.com, pana.petrovic@gmail.com}

% Se comentar a linha abaixo, irá aparecer a data quando foi compilada a apresentação  
\date{15. maj 2019}

\pgfdeclareimage[height=1cm]{inf}{figs/CienciaDaComputacao.png}

% pode-se colocar o LOGO assim
\logo{\pgfuseimage{inf}}

%\AtBeginSection[]{
 % \begin{frame}<beamer>
 %   \frametitle{Pregled}
  %  \tableofcontents[currentsection,currentsubsection]
 % \end{frame}
%}

\begin{document}

\begin{frame}
\titlepage
\end{frame}

\begin{frame}
\frametitle{Pregled}
\tableofcontents
\end{frame}



\section{\textbf {Pojam privatnosti}}
\frame{
	\frametitle{Pojam privatnosti}
	Šta je privatnost?
	\begin{itemize}
		\item Širok i neodređen pojam 
		\begin{itemize}
			\item Fizička bliskost osobi
			\item Znanje o osobi
			\item Sukob privatnog i javnog
			\newline
		\end{itemize}
	
		\item \textbf{Koncept društvenog ugovora o kontroli pristupa podacima i telu pojedinca}
	\newline
		\item Percepcija privatnosti
		\begin{itemize}
			\item Kako pojedinac vidi svoju privatnost
			\item Zavisi od pola, uzrasta, obrazovanja, prethodnog iskustva
			\item Ponašanje pojedinca na internetu
			\item Odricanje privatnosti zarad ispunjenja potreba
			
		\end{itemize}
		
	\end{itemize}
}

\section{\textbf {Pojedinac na internetu}}
\subsection{Društvene mreže}
\frame{

\frametitle{Društvene mreže}
	
	\begin{itemize}
		\item Dve trećine korisnika interneta 
		\item Podešavanja privatnosti prepuštena korisniku
		\item Da li je to garancija privatnosti?
		\item Pretežno mladi ljudi
	\end{itemize}
		
\begin{table}[h!]
	\begin{center}
		\caption{Broj tinejdžera koji su delili svoje podatke na društvenim mrežama u procentima}
		\begin{tabular}{|c | c c c c c|} \hline
			
			Godina & \begin{tabular}[x]{@{}c@{}} Sopstvena \\fotografija \end{tabular} & \begin{tabular}[x]{@{}c@{}} Naziv \\ škole \end{tabular} & \begin{tabular}[x]{@{}c@{}}Mesto \\ stanovanja \end{tabular} & \begin{tabular}[x]{@{}c@{}}Imejl \\ adresa \end{tabular} & \begin{tabular}[x]{@{}c@{}}Broj \\ telefona \end{tabular} \\
			\hline
			2006 & 79 & 49 & 61 & 29 & 2 \\
			2012 & 91 & 71 & 71 & 53 & 20 \\
			\hline
		\end{tabular}
		\label{tab:tabela1}
	\end{center}
\end{table}
}

\subsection{Krađa identiteta}

\frame{
	\frametitle{Krađa identiteta}
	\begin{itemize}
		\item Fenomen u porastu
		\item Spam mejlovi i automatski procesi
		\newline
		\item Deepfake tehnologija
		\begin{itemize}
			\item Manipulacija video sadržajem 
			\item Algoritmi za mašinsko učenje
			\item Novi stepen krađe identiteta 
		\end{itemize}
		
	\end{itemize}

	\pgfdeclareimage[width=7.5cm]{D}{deepfake.png}
	\begin{center}
	\pgfuseimage{D}	
	\end{center}
	


}

\subsection{Kolačići}

\frame{
	\frametitle{Kolačići}
	\begin{itemize}
		\item Male datoteke koje čuvaju informacije o veb sajtovima
	
		\item Sesijski i trajni 
		\item Kolačići prve strane - korisni kolačići
		\item Kolačići treće strane - kolačići za praćenje 
		\begin{itemize}
			\item Istorija pretraživanja, kupovine, lokacije, itd.
			\item Korisnici retko čitaju politike privatnosti
			\item Koristiti privatni režim ili deaktivirati kolačiće
		\end{itemize}	

	\end{itemize}

	\pgfdeclareimage[width=11cm]{M}{henkelkolacic.png}
	\pgfuseimage{M}	
}



\section{\textbf {Zaštita privatnosti na internetu}}
\frame{
    \frametitle{Mere preduzimanja zaštite na internetu}
    
    \pgfdeclareimage[height=6.5cm]{RM}{chart.png}
    \pgfuseimage{RM}    
}

\subsection{Virtuelne privatne mreže}
\frame{
	\frametitle{VPN}
	\pgfdeclareimage[width=11cm]{R}{vpn.jpg}
	\pgfuseimage{R} 
	\begin{itemize}
		\item Zaštita od hakovanja i gubitka podataka
		\item Zaštita privatnosti i od cenzurisanja
		\item Samo članovi mreže mogu da vide poslate informacije
		\item Sigurnosne procedure (enkripcija) i protokoli tuneliranja
		\item Sakriva IP adresu
	\end{itemize}
}

\subsection{Enkripcija}
\frame{
	\frametitle{Enkripcija}
	\begin{itemize}
		\item Problem posredovane komunikacije udaljenih uređaja
		\item Mehanizam za šifrovanje informacija korišćenjem ključa
		\begin{itemize}
			\item Učitavanje https stranica
			\item VPN
			\item Višeslojna enkripcija - Tor pregledač
			\item Od-početka-do-kraja - aplikacije za ćaskanje
		\end{itemize}		
	\end{itemize}
	
	\pgfdeclareimage[width=6.5cm]{E}{enc2.png}
	\begin{center}
		\pgfuseimage{E} 
	\end{center}
}


\section{\textbf {Zaključak}}

\frame{
	\frametitle{Zaključak}
	
	\begin{itemize}
		\item Konstantan rizik od zloupotrebe informacija
		\item Privatnost na internetu danas?
		\begin{itemize}
			\item Postoji, ali nije zagarantovana
			\item Aktivno angažovanje i napor korisnika 
			\newline  
		\end{itemize}
		\item Odreći se ili zadržati privatnost? 
		\item Rešenje u informisanju pojedinca  
	\\~\\~\\ 
\end{itemize}
	\begin{center}
	    \color{aqua}\textbf{\huge {Hvala na pažnji!}}
	\end{center}
	
}

\section{\textbf{Literatura}}

\frame{
    \frametitle{Literatura}
    \begin{itemize}
	\begin{tiny}
	%	https://thenextweb.com/artificial-intelligence/2018/02/21/deepfakes-algorithm-nails-donald-trump-in-most-convincing-fake-yet/
	   \item D. Anon, \href{https://privacy.net/stop-cookies-tracking/}{\color{blue}{How cookies track you around the web and how to stop them}}, 2018.
	   \item https://www.cnet.com/best-vpn-services-directory/
	   \item K. Grewal, R. Kajal, and D. Saini. Virtual Private Network. International Journal of Advanced Research in Computer Science and Software Engineering, 2012. on-line at: www.ijarcsse.com.
	   \item N. Lord. What Is Data Encryption? Definition, Best Practices \& More. Digital Guardian, 2019. 
       \item M. Madden, A. Lenhart, S. Cortesi, U. Gasser, M. Duggan, A. Smith and M. Beaton. Teens, Social media, and Privacy. Pew Research Center, 2013.
       \item M. Madden and L. Rainie. Americans' Attitudes About Privacy, Security and Surveillance. Pew Research Center, 2015.
       \item R. Mekovec. Online privacy: overview and preliminary research. Journal of Information and Organizational Sciences, pages 195-209, 2010.
       \item V. Perta. A Glance through the VPN Looking Glass: IPv6 Leakage and DNS Hijacking in Commercial VPN clients. 2015.
       \item M. J. Quinn. Ethics for the Information Age, chapter Information Privacy, pages 227-314. AddisonWesley Professional, Boston, 2014.
	   \item J. Y. Tsai, S. Egelman, L. Cranor and A. Acquisti. The Effect of Online Privacy Information on Purchasing Behavior: An Experimental Study. Information Systems Reseach, 2011.
	   \item
	\end{tiny}

	\end{itemize}
}

\end{document}